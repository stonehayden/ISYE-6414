% Options for packages loaded elsewhere
\PassOptionsToPackage{unicode}{hyperref}
\PassOptionsToPackage{hyphens}{url}
%
\documentclass[
]{article}
\title{HW1 Peer Assessment}
\author{}
\date{\vspace{-2.5em}}

\usepackage{amsmath,amssymb}
\usepackage{lmodern}
\usepackage{iftex}
\ifPDFTeX
  \usepackage[T1]{fontenc}
  \usepackage[utf8]{inputenc}
  \usepackage{textcomp} % provide euro and other symbols
\else % if luatex or xetex
  \usepackage{unicode-math}
  \defaultfontfeatures{Scale=MatchLowercase}
  \defaultfontfeatures[\rmfamily]{Ligatures=TeX,Scale=1}
\fi
% Use upquote if available, for straight quotes in verbatim environments
\IfFileExists{upquote.sty}{\usepackage{upquote}}{}
\IfFileExists{microtype.sty}{% use microtype if available
  \usepackage[]{microtype}
  \UseMicrotypeSet[protrusion]{basicmath} % disable protrusion for tt fonts
}{}
\makeatletter
\@ifundefined{KOMAClassName}{% if non-KOMA class
  \IfFileExists{parskip.sty}{%
    \usepackage{parskip}
  }{% else
    \setlength{\parindent}{0pt}
    \setlength{\parskip}{6pt plus 2pt minus 1pt}}
}{% if KOMA class
  \KOMAoptions{parskip=half}}
\makeatother
\usepackage{xcolor}
\IfFileExists{xurl.sty}{\usepackage{xurl}}{} % add URL line breaks if available
\IfFileExists{bookmark.sty}{\usepackage{bookmark}}{\usepackage{hyperref}}
\hypersetup{
  pdftitle={HW1 Peer Assessment},
  hidelinks,
  pdfcreator={LaTeX via pandoc}}
\urlstyle{same} % disable monospaced font for URLs
\usepackage[margin=1in]{geometry}
\usepackage{color}
\usepackage{fancyvrb}
\newcommand{\VerbBar}{|}
\newcommand{\VERB}{\Verb[commandchars=\\\{\}]}
\DefineVerbatimEnvironment{Highlighting}{Verbatim}{commandchars=\\\{\}}
% Add ',fontsize=\small' for more characters per line
\usepackage{framed}
\definecolor{shadecolor}{RGB}{248,248,248}
\newenvironment{Shaded}{\begin{snugshade}}{\end{snugshade}}
\newcommand{\AlertTok}[1]{\textcolor[rgb]{0.94,0.16,0.16}{#1}}
\newcommand{\AnnotationTok}[1]{\textcolor[rgb]{0.56,0.35,0.01}{\textbf{\textit{#1}}}}
\newcommand{\AttributeTok}[1]{\textcolor[rgb]{0.77,0.63,0.00}{#1}}
\newcommand{\BaseNTok}[1]{\textcolor[rgb]{0.00,0.00,0.81}{#1}}
\newcommand{\BuiltInTok}[1]{#1}
\newcommand{\CharTok}[1]{\textcolor[rgb]{0.31,0.60,0.02}{#1}}
\newcommand{\CommentTok}[1]{\textcolor[rgb]{0.56,0.35,0.01}{\textit{#1}}}
\newcommand{\CommentVarTok}[1]{\textcolor[rgb]{0.56,0.35,0.01}{\textbf{\textit{#1}}}}
\newcommand{\ConstantTok}[1]{\textcolor[rgb]{0.00,0.00,0.00}{#1}}
\newcommand{\ControlFlowTok}[1]{\textcolor[rgb]{0.13,0.29,0.53}{\textbf{#1}}}
\newcommand{\DataTypeTok}[1]{\textcolor[rgb]{0.13,0.29,0.53}{#1}}
\newcommand{\DecValTok}[1]{\textcolor[rgb]{0.00,0.00,0.81}{#1}}
\newcommand{\DocumentationTok}[1]{\textcolor[rgb]{0.56,0.35,0.01}{\textbf{\textit{#1}}}}
\newcommand{\ErrorTok}[1]{\textcolor[rgb]{0.64,0.00,0.00}{\textbf{#1}}}
\newcommand{\ExtensionTok}[1]{#1}
\newcommand{\FloatTok}[1]{\textcolor[rgb]{0.00,0.00,0.81}{#1}}
\newcommand{\FunctionTok}[1]{\textcolor[rgb]{0.00,0.00,0.00}{#1}}
\newcommand{\ImportTok}[1]{#1}
\newcommand{\InformationTok}[1]{\textcolor[rgb]{0.56,0.35,0.01}{\textbf{\textit{#1}}}}
\newcommand{\KeywordTok}[1]{\textcolor[rgb]{0.13,0.29,0.53}{\textbf{#1}}}
\newcommand{\NormalTok}[1]{#1}
\newcommand{\OperatorTok}[1]{\textcolor[rgb]{0.81,0.36,0.00}{\textbf{#1}}}
\newcommand{\OtherTok}[1]{\textcolor[rgb]{0.56,0.35,0.01}{#1}}
\newcommand{\PreprocessorTok}[1]{\textcolor[rgb]{0.56,0.35,0.01}{\textit{#1}}}
\newcommand{\RegionMarkerTok}[1]{#1}
\newcommand{\SpecialCharTok}[1]{\textcolor[rgb]{0.00,0.00,0.00}{#1}}
\newcommand{\SpecialStringTok}[1]{\textcolor[rgb]{0.31,0.60,0.02}{#1}}
\newcommand{\StringTok}[1]{\textcolor[rgb]{0.31,0.60,0.02}{#1}}
\newcommand{\VariableTok}[1]{\textcolor[rgb]{0.00,0.00,0.00}{#1}}
\newcommand{\VerbatimStringTok}[1]{\textcolor[rgb]{0.31,0.60,0.02}{#1}}
\newcommand{\WarningTok}[1]{\textcolor[rgb]{0.56,0.35,0.01}{\textbf{\textit{#1}}}}
\usepackage{longtable,booktabs,array}
\usepackage{calc} % for calculating minipage widths
% Correct order of tables after \paragraph or \subparagraph
\usepackage{etoolbox}
\makeatletter
\patchcmd\longtable{\par}{\if@noskipsec\mbox{}\fi\par}{}{}
\makeatother
% Allow footnotes in longtable head/foot
\IfFileExists{footnotehyper.sty}{\usepackage{footnotehyper}}{\usepackage{footnote}}
\makesavenoteenv{longtable}
\usepackage{graphicx}
\makeatletter
\def\maxwidth{\ifdim\Gin@nat@width>\linewidth\linewidth\else\Gin@nat@width\fi}
\def\maxheight{\ifdim\Gin@nat@height>\textheight\textheight\else\Gin@nat@height\fi}
\makeatother
% Scale images if necessary, so that they will not overflow the page
% margins by default, and it is still possible to overwrite the defaults
% using explicit options in \includegraphics[width, height, ...]{}
\setkeys{Gin}{width=\maxwidth,height=\maxheight,keepaspectratio}
% Set default figure placement to htbp
\makeatletter
\def\fps@figure{htbp}
\makeatother
\setlength{\emergencystretch}{3em} % prevent overfull lines
\providecommand{\tightlist}{%
  \setlength{\itemsep}{0pt}\setlength{\parskip}{0pt}}
\setcounter{secnumdepth}{-\maxdimen} % remove section numbering
\ifLuaTeX
  \usepackage{selnolig}  % disable illegal ligatures
\fi

\begin{document}
\maketitle

\hypertarget{part-a.-anova}{%
\section{Part A. ANOVA}\label{part-a.-anova}}

Additional Material: ANOVA tutorial

\url{https://datascienceplus.com/one-way-anova-in-r/}

Jet lag is a common problem for people traveling across multiple time
zones, but people can gradually adjust to the new time zone since the
exposure of the shifted light schedule to their eyes can resets the
internal circadian rhythm in a process called ``phase shift''. Campbell
and Murphy (1998) in a highly controversial study reported that the
human circadian clock can also be reset by only exposing the back of the
knee to light, with some hailing this as a major discovery and others
challenging aspects of the experimental design. The table below is taken
from a later experiment by Wright and Czeisler (2002) that re-examined
the phenomenon. The new experiment measured circadian rhythm through the
daily cycle of melatonin production in 22 subjects randomly assigned to
one of three light treatments. Subjects were woken from sleep and for
three hours were exposed to bright lights applied to the eyes only, to
the knees only or to neither (control group). The effects of treatment
to the circadian rhythm were measured two days later by the magnitude of
phase shift (measured in hours) in each subject's daily cycle of
melatonin production. A negative measurement indicates a delay in
melatonin production, a predicted effect of light treatment, while a
positive number indicates an advance.

Raw data of phase shift, in hours, for the circadian rhythm experiment

\begin{longtable}[]{@{}ll@{}}
\toprule
Treatment & Phase Shift (hr) \\
\midrule
\endhead
Control & 0.53, 0.36, 0.20, -0.37, -0.60, -0.64, -0.68, -1.27 \\
Knees & 0.73, 0.31, 0.03, -0.29, -0.56, -0.96, -1.61 \\
Eyes & -0.78, -0.86, -1.35, -1.48, -1.52, -2.04, -2.83 \\
\bottomrule
\end{longtable}

\hypertarget{question-a1---3-pts}{%
\subsection{Question A1 - 3 pts}\label{question-a1---3-pts}}

Consider the following incomplete R output:

\begin{longtable}[]{@{}cccccc@{}}
\toprule
Source & Df & Sum of Squares & Mean Squares & F-statistics & p-value \\
\midrule
\endhead
Treatments & ? & ? & 3.6122 & ? & 0.004 \\
Error & ? & 9.415 & ? & & \\
TOTAL & ? & ? & & & \\
\bottomrule
\end{longtable}

Fill in the missing values in the analysis of the variance table.Note:
Missing values can be calculated using the corresponding formulas
provided in the lectures, or you can build the data frame in R and
generate the ANOVA table using the aov() function. Either approach will
be accepted.

\hypertarget{question-a2---3-pts}{%
\subsection{Question A2 - 3 pts}\label{question-a2---3-pts}}

Use \(\mu_1\), \(\mu_2\), and \(\mu_3\) as notation for the three mean
parameters and define these parameters clearly based on the context of
the topic above (i.e.~explain what \(\mu_1\), \(\mu_2\), and \(\mu_3\)
mean in words in the context of this problem). Find the estimates of
these parameters.

\hypertarget{question-a3---5-pts}{%
\subsection{Question A3 - 5 pts}\label{question-a3---5-pts}}

Use the ANOVA table in Question A1 to answer the following questions:

\begin{enumerate}
\def\labelenumi{\alph{enumi}.}
\item
  \textbf{1 pts} Write the null hypothesis of the ANOVA \(F\)-test,
  \(H_0\)
\item
  \textbf{1 pts} Write the alternative hypothesis of the ANOVA
  \(F\)-test, \(H_A\)
\item
  \textbf{1 pts} Fill in the blanks for the degrees of freedom of the
  ANOVA \(F\)-test statistic: \(F\)(\_\_\_\_, \_\_\_\_\_)
\item
  \textbf{1 pts} What is the p-value of the ANOVA \(F\)-test?
\item
  \textbf{1 pts} According the the results of the ANOVA \(F\)-test, does
  light treatment affect phase shift? Use an \(\alpha\)-level of 0.05.
\end{enumerate}

\hypertarget{part-b.-simple-linear-regression}{%
\section{Part B. Simple Linear
Regression}\label{part-b.-simple-linear-regression}}

We are going to use regression analysis to estimate the performance of
CPUs based on the maximum number of channels in the CPU. This data set
comes from the UCI Machine Learning Repository.

The data file includes the following columns:

\begin{itemize}
\tightlist
\item
  \emph{vendor}: vendor of the CPU
\item
  \emph{chmax}: maximum channels in the CPU
\item
  \emph{performance}: published relative performance of the CPU
\end{itemize}

The data is in the file ``machine.csv''. To read the data in \texttt{R},
save the file in your working directory (make sure you have changed the
directory if different from the R working directory) and read the data
using the \texttt{R} function \texttt{read.csv()}.

\begin{Shaded}
\begin{Highlighting}[]
\CommentTok{\# Read in the data}
\NormalTok{data }\OtherTok{=} \FunctionTok{read.csv}\NormalTok{(}\StringTok{"machine.csv"}\NormalTok{, }\AttributeTok{head =} \ConstantTok{TRUE}\NormalTok{, }\AttributeTok{sep =} \StringTok{","}\NormalTok{)}
\CommentTok{\# Show the first few rows of data}
\FunctionTok{head}\NormalTok{(data, }\DecValTok{3}\NormalTok{)}
\end{Highlighting}
\end{Shaded}

\begin{verbatim}
##    vendor chmax performance
## 1 adviser   128         198
## 2  amdahl    32         269
## 3  amdahl    32         220
\end{verbatim}

\hypertarget{question-b1-exploratory-data-analysis---9-pts}{%
\subsection{Question B1: Exploratory Data Analysis - 9
pts}\label{question-b1-exploratory-data-analysis---9-pts}}

\begin{enumerate}
\def\labelenumi{\alph{enumi}.}
\tightlist
\item
  \textbf{3 pts} Use a scatter plot to describe the relationship between
  CPU performance and the maximum number of channels. Describe the
  general trend (direction and form). Include plots and R-code used.
\end{enumerate}

\begin{Shaded}
\begin{Highlighting}[]
\CommentTok{\# Your code here...}
\end{Highlighting}
\end{Shaded}

\begin{enumerate}
\def\labelenumi{\alph{enumi}.}
\setcounter{enumi}{1}
\tightlist
\item
  \textbf{3 pts} What is the value of the correlation coefficient
  between \emph{performance} and \emph{chmax}? Please interpret the
  strength of the correlation based on the correlation coefficient.
\end{enumerate}

\begin{Shaded}
\begin{Highlighting}[]
\CommentTok{\# Your code here...}
\end{Highlighting}
\end{Shaded}

\begin{enumerate}
\def\labelenumi{\alph{enumi}.}
\setcounter{enumi}{2}
\item
  \textbf{2 pts} Based on this exploratory analysis, would you recommend
  a simple linear regression model for the relationship?
\item
  \textbf{1 pts} Based on the analysis above, would you pursue a
  transformation of the data? \emph{Do not transform the data.}
\end{enumerate}

\hypertarget{question-b2-fitting-the-simple-linear-regression-model---11-pts}{%
\subsection{Question B2: Fitting the Simple Linear Regression Model - 11
pts}\label{question-b2-fitting-the-simple-linear-regression-model---11-pts}}

Fit a linear regression model, named \emph{model1}, to evaluate the
relationship between performance and the maximum number of channels.
\emph{Do not transform the data.} The function you should use in R is:

\begin{Shaded}
\begin{Highlighting}[]
\CommentTok{\# Your code here...}
\NormalTok{model1 }\OtherTok{=} \FunctionTok{lm}\NormalTok{(performance }\SpecialCharTok{\textasciitilde{}}\NormalTok{ chmax, data)}
\end{Highlighting}
\end{Shaded}

\begin{enumerate}
\def\labelenumi{\alph{enumi}.}
\item
  \textbf{3 pts} What are the model parameters and what are their
  estimates?
\item
  \textbf{2 pts} Write down the estimated simple linear regression
  equation.
\item
  \textbf{2 pts} Interpret the estimated value of the \(\beta_1\)
  parameter in the context of the problem.
\item
  \textbf{2 pts} Find a 95\% confidence interval for the \(\beta_1\)
  parameter. Is \(\beta_1\) statistically significant at this level?
\item
  \textbf{2 pts} Is \(\beta_1\) statistically significantly positive at
  an \(\alpha\)-level of 0.01? What is the approximate p-value of this
  test?
\end{enumerate}

\hypertarget{question-b3-checking-the-assumptions-of-the-model---8-pts}{%
\subsection{Question B3: Checking the Assumptions of the Model - 8
pts}\label{question-b3-checking-the-assumptions-of-the-model---8-pts}}

Create and interpret the following graphs with respect to the
assumptions of the linear regression model. In other words, comment on
whether there are any apparent departures from the assumptions of the
linear regression model. Make sure that you state the model assumptions
and assess each one. Each graph may be used to assess one or more model
assumptions.

\begin{enumerate}
\def\labelenumi{\alph{enumi}.}
\tightlist
\item
  \textbf{2 pts} Scatterplot of the data with \emph{chmax} on the x-axis
  and \emph{performance} on the y-axis
\end{enumerate}

\begin{Shaded}
\begin{Highlighting}[]
\CommentTok{\# Your code here...}
\end{Highlighting}
\end{Shaded}

\textbf{Model Assumption(s) it checks:}

\textbf{Interpretation:}

\begin{enumerate}
\def\labelenumi{\alph{enumi}.}
\setcounter{enumi}{1}
\tightlist
\item
  \textbf{3 pts} Residual plot - a plot of the residuals,
  \(\hat\epsilon_i\), versus the fitted values, \(\hat{y}_i\)
\end{enumerate}

\begin{Shaded}
\begin{Highlighting}[]
\CommentTok{\# Your code here...}
\end{Highlighting}
\end{Shaded}

\textbf{Model Assumption(s) it checks:}

\textbf{Interpretation:}

\begin{enumerate}
\def\labelenumi{\alph{enumi}.}
\setcounter{enumi}{2}
\tightlist
\item
  \textbf{3 pts} Histogram and q-q plot of the residuals
\end{enumerate}

\begin{Shaded}
\begin{Highlighting}[]
\CommentTok{\# Your code here...}
\end{Highlighting}
\end{Shaded}

\textbf{Model Assumption(s) it checks:}

\textbf{Interpretation:}

\hypertarget{question-b4-improving-the-fit---10-pts}{%
\subsection{Question B4: Improving the Fit - 10
pts}\label{question-b4-improving-the-fit---10-pts}}

\begin{enumerate}
\def\labelenumi{\alph{enumi}.}
\tightlist
\item
  \textbf{2 pts} Use a Box-Cox transformation (\texttt{boxCox()}) in
  \texttt{car()} package or (\texttt{boxcox()}) in \texttt{MASS()}
  package to find the optimal \(\lambda\) value rounded to the nearest
  half integer. What transformation of the response, if any, does it
  suggest to perform?
\end{enumerate}

\begin{Shaded}
\begin{Highlighting}[]
\CommentTok{\# Your code here...}
\end{Highlighting}
\end{Shaded}

\begin{enumerate}
\def\labelenumi{\alph{enumi}.}
\setcounter{enumi}{1}
\tightlist
\item
  \textbf{2 pts} Create a linear regression model, named \emph{model2},
  that uses the log transformed \emph{performance} as the response, and
  the log transformed \emph{chmax} as the predictor. Note: The variable
  \emph{chmax} has a couple of zero values which will cause problems
  when taking the natural log. Please add one to the predictor before
  taking the natural log of it
\end{enumerate}

\begin{Shaded}
\begin{Highlighting}[]
\CommentTok{\# Your code here...}
\end{Highlighting}
\end{Shaded}

\begin{enumerate}
\def\labelenumi{\alph{enumi}.}
\setcounter{enumi}{2}
\item
  \textbf{2 pts} Compare the R-squared values of \emph{model1} and
  \emph{model2}. Did the transformation improve the explanatory power of
  the model?
\item
  \textbf{4 pts} Similar to Question B3, assess and interpret all model
  assumptions of \emph{model2}. A model is considered a good fit if all
  assumptions hold. Based on your interpretation of the model
  assumptions, is \emph{model2} a good fit?
\end{enumerate}

\begin{Shaded}
\begin{Highlighting}[]
\CommentTok{\# Your code here...}
\end{Highlighting}
\end{Shaded}

\hypertarget{question-b5-prediction---3-pts}{%
\subsection{Question B5: Prediction - 3
pts}\label{question-b5-prediction---3-pts}}

Suppose we are interested in predicting CPU performance when
\texttt{chmax\ =\ 128}. Please make a prediction using both
\emph{model1} and \emph{model2} and provide the 95\% prediction interval
of each prediction on the original scale of the response,
\emph{performance}. What observations can you make about the result in
the context of the problem?

\begin{Shaded}
\begin{Highlighting}[]
\CommentTok{\# Your code here...}
\end{Highlighting}
\end{Shaded}

\hypertarget{part-c.-anova---8-pts}{%
\section{Part C. ANOVA - 8 pts}\label{part-c.-anova---8-pts}}

We are going to continue using the CPU data set to analyse various
vendors in the data set. There are over 20 vendors in the data set. To
simplify the task, we are going to limit our analysis to three vendors,
specifically, honeywell, hp, and nas. The code to filter for those
vendors is provided below.

\begin{Shaded}
\begin{Highlighting}[]
\CommentTok{\# Filter for honeywell, hp, and nas}
\NormalTok{data2 }\OtherTok{=}\NormalTok{ data[data}\SpecialCharTok{$}\NormalTok{vendor }\SpecialCharTok{\%in\%} \FunctionTok{c}\NormalTok{(}\StringTok{"honeywell"}\NormalTok{, }\StringTok{"hp"}\NormalTok{, }\StringTok{"nas"}\NormalTok{), ]}
\NormalTok{data2}\SpecialCharTok{$}\NormalTok{vendor }\OtherTok{=} \FunctionTok{factor}\NormalTok{(data2}\SpecialCharTok{$}\NormalTok{vendor)}
\end{Highlighting}
\end{Shaded}

\begin{enumerate}
\def\labelenumi{\arabic{enumi}.}
\tightlist
\item
  \textbf{2 pts} Using \texttt{data2}, create a boxplot of
  \emph{performance} and \emph{vendor}, with \emph{performance} on the
  vertical axis. Interpret the plots.
\end{enumerate}

\begin{Shaded}
\begin{Highlighting}[]
\CommentTok{\# Your code here...}
\end{Highlighting}
\end{Shaded}

\begin{enumerate}
\def\labelenumi{\arabic{enumi}.}
\setcounter{enumi}{1}
\tightlist
\item
  \textbf{3 pts} Perform an ANOVA F-test on the means of the three
  vendors. Using an \(\alpha\)-level of 0.05, can we reject the null
  hypothesis that the means of the three vendors are equal? Please
  interpret.
\end{enumerate}

\begin{Shaded}
\begin{Highlighting}[]
\CommentTok{\# Your code here...}
\end{Highlighting}
\end{Shaded}

\begin{enumerate}
\def\labelenumi{\arabic{enumi}.}
\setcounter{enumi}{2}
\tightlist
\item
  \textbf{3 pts} Perform a Tukey pairwise comparison between the three
  vendors. Using an \(\alpha\)-level of 0.05, which means are
  statistically significantly different from each other?
\end{enumerate}

\begin{Shaded}
\begin{Highlighting}[]
\CommentTok{\# Your code here...}
\end{Highlighting}
\end{Shaded}


\end{document}
